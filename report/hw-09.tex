\documentclass[fleqn,11pt]{article}
\usepackage[includeheadfoot,margin=0.5in,headheight=18pt]{geometry}
\usepackage[english]{babel}
% \usepackage{firamath-otf}
% \usepackage{cmbright}
% \usepackage[OT1]{fontenc}
\usepackage[sfdefault]{notomath}
\usepackage{calc}
\usepackage{booktabs}
\usepackage{multirow}
\usepackage{multicol}
\usepackage[table,xcdraw]{xcolor}
\usepackage[version=4]{mhchem}
\usepackage[sticky-per,print-unity-mantissa=false]{siunitx}
\usepackage{cancel}
\usepackage{contour}
\usepackage{ulem}
\usepackage{enumitem}
\usepackage{listings}
\usepackage{arydshln}
\usepackage{pdfpages}
\usepackage{fancyhdr}
\usepackage{lastpage}
\usepackage{wrapfig}
\usepackage{makecell}
\usepackage{float}
\usepackage{tikz}
\usepackage{pgfplots}
\usepackage{pgfplotstable}
\usepackage{circuitikz}
\usepackage{tikzpagenodes}
% \usepackage{tabularx}
\usepackage{hyperref}
\usepackage{graphicx}
\usepackage{caption}
\usepackage{subcaption}
\usepackage{parskip}
\usepackage{setspace}
\usepackage{cleveref}
% \setoperatorfont\mathsf
% \renewcommand*\familydefault{\sfdefault}

\frenchspacing
\renewcommand{\ULdepth}{1.8pt}
\contourlength{0.8pt}

\usetikzlibrary{positioning}
\usepgfplotslibrary{external}
% \tikzexternalize{}

\pgfplotsset{compat=1.18}
\def\axisdefaultwidth{5in}

\lstset{
  basicstyle=\ttfamily\footnotesize,
  breaklines=true,
  breakatwhitespace=false,
  numbers=left,
  keywordstyle=\pmb,
  numberstyle=\tiny,
  commentstyle=\itshape,
  showstringspaces=false,
}

\renewcommand{\title}{Homework 9} % TODO change this

\hypersetup{
  colorlinks=true,
  linkcolor=blue,
  filecolor=magenta,      
  urlcolor=cyan,
  pdftitle={\title},
  pdfpagemode=FullScreen,
}

\pagestyle{fancyplain}
\lhead{\Large \textbf{\title}}
\rhead{\large Sumit Basak}
\lfoot{NERS 561}
\cfoot{}
\rfoot{Page \thepage~of~\pageref{LastPage}}

\newcommand{\md}[2][d]{\ensuremath{\operatorname{#1}\!{#2}}}
\newcommand{\mder}[3][d]{\frac{\md[#1]{#2}}{\md[#1]{#3}}}
\newcommand{\textarrow}[2]{\xRightarrow{\substack{\text{#1} \\ \text{#2}}}}
\renewcommand{\vec}[1]{\ensuremath{\mathbf{#1}}}

\sisetup{exponent-product=*}
\colorlet{darkgreen}{green!50!black}

\DeclareSIUnit{\year}{yr}
\DeclareSIUnit{\curie}{Ci}
\DeclareSIUnit{\calorie}{cal}
\DeclareSIUnit{\atom}{atom}
\DeclareSIUnit{\df}{\SIUnitSymbolDegree F}
\DeclareSIUnit{\btu}{BTU}
\DeclareSIUnit{\lbm}{lbm}
\DeclareSIUnit{\inch}{in}
\DeclareSIUnit{\atom}{atoms}
\DeclareSIUnit{\barn}{b}

\newcommand{\textover}[3][c]{%
  % #1 is the alignment, default c
  % #2 is the text to be printed
  % #3 is the text for setting the width
  \makebox[\widthof{#3}][#1]{#2}%
}

\newcommand{\myuline}[1]{%
  \uline{\phantom{#1}}%
  \llap{\contour{white}{#1}}%
}

\begin{document}
\section{Implementation}
The code for this assignment is attached in appendix. It is 
split into seven regions, marked with \texttt{\#region 
\(\dots\) \#endregion} statements. A description of each 
region follows. \begin{description}
  \item [Input] Parses the given input file for the required parameters.

  \item [Mesh Balance Equations] Sets up four vectors, \(a\) through \(d\), with the following contents, where \(i\) is the mesh index.
    \begin{align*}
      a_i &= -\tilde D_{i - 1} \\
      b_i &= \Sigma_{a, i} \md{h} - a_i - c_i \\
      c_i &= -\tilde D_i \\
      d_i &= \nu \Sigma_{f, i} \md{h}
      \intertext{where,}
      \tilde D_i &= \frac{1}{\frac{\md{h}}{2 D_i} + \frac{\md{h}}{2 D_{i + 1}}}
    \end{align*}
  
  \item [LU Factorization] Splits the transformation matrix \(M\) into \(L\) and \(U\) by creating vectors \(\tilde a\) and \(\tilde b\) as follows.
    Also prepares \(\psi = \lambda F \phi\), the fission source in each mesh region.
    \begin{align*}
      \tilde a &= \frac{a_i}{\tilde b_{i - 1}} \\
      \tilde b &= b_i - \tilde a_i c_{i - 1}
      \intertext{The matricies are defined as,}
      L &= \begin{bmatrix}
        1 & 0 & 0 & 0 & 0 \\
        \tilde a_2 & 1 & \ddots & \ddots & 0 \\
        0 & \ddots & \ddots & \ddots & 0 \\
        0 & \ddots & \ddots & \ddots & 0 \\
        0 & 0 & 0 & \tilde a_N & 1
      \end{bmatrix} \\
      U &= \begin{bmatrix}
        \tilde b_1 & c_1 & 0 & 0 & 0 \\
        0 & \tilde b_2 & \ddots & \ddots & 0 \\
        0 & \ddots & \ddots & \ddots & 0 \\
        0 & \ddots & \ddots & \ddots & c_{N - 1} \\
        0 & 0 & 0 & 0 & \tilde b_N
      \end{bmatrix}
    \end{align*}
  
  \item [Forward Elimination] Obtains a temporary vector \(y\) by solving \(L y = \psi\) through forward elimination.
  
  \item [Backward Substitution] Obtains the flux by solving \(U \phi = y\) by backward substitution.
  
  \item [Outer Iteration] Checks for convergence in the flux \(\phi\) and eigenvalue \(\lambda\) and loops if needed.
  
  \item [Output] Normalizes flux and fission source and prints them to a file.
\end{description}

\section{Results}
We are told to solve find the flux and fission source given 
the following input file. This input specifies a \SI{180}{\centi\meter}
problem domain with nine regions and five meshes per region.
There are four possible composition each region can take,
and their mappings are provided on Line~10. Tolerance for 
\(k\) is \num{e-6}, while that of flux is \num{e-5}. We use 
the infinity norm -- that is, the maximum absolute difference 
among the elements of the vector -- for the latter. Results 
are plotted below.

\lstinputlisting{../input.txt}

The resulting eigenvalue is \(k = 0.87309\), with a dominance
ratio \(\sigma = 0.96754\). This solution took 175 iterations
to converge.

\begin{center}
  \begin{tikzpicture}[
    trim axis left,
    trim axis right,
  ]
    \begin{axis}[
      xlabel={\(x\) {[\si{\centi\meter}]}},
      ylabel={Normalized Value},
      legend pos={outer north east},
      legend cell align={left}
    ]
      \addplot[mark=none,const plot,solid] table {../phi.txt};
      \addplot[mark=none,const plot,dotted] table {../source.txt};

      \addlegendentry{Flux}
      \addlegendentry{Fission Source}
    \end{axis}
  \end{tikzpicture}
\end{center}

\newpage \appendix 
\section{Source Code}
The full source code for both the solver and this report is 
available at \url{https://github.com/mittsq/ners-561-hw9}; 
I made sure to make the repository public this time. The 
important parts of the code are included below.

\lstinputlisting[language={[Sharp]C}]{../code/Program.cs}
\end{document}
